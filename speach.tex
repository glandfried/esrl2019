
I am a social anthropologist, following a PhD in computational science. 

%

My work is focused on cultural evolution and social learning, which I developed through an intensive use of programming and statistics tools.
We use databases from virtual communities that I create from information available on Internet, and I analyze them with Bayesian Inference and others statistical methods.
This is the main structure of my current lines of works.

%

In the last paper, we ask about the effects of social behavior on individual learning curves.
We rely on a multiplayer online game in which games could be played individually or in teams.
We found that team-oriented players standout, but stable teammates has the most powerful skill boost during the first games of experience, that individually would only be acquired after thousands of games.
We use the state-of-art skill ranking system developed by Microsoft.
Relying only on the results of games, the model infer individual skill.
The skill difference between players is a good predictor of how would win, so we can trust on it.

% 

The model assumes that players have, at a given moment, a certain  skill, but when they play they turn out a performance that is around their true skill.
The model assumes that the observed winner is the player who provide the highest performance.
The extended model allows estimating individual skill from team games.

